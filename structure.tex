%%%%%%%%%%%%%%%%%%%%%%%%%%%%%%%%%%%%%%%%%
% Kristoffer Resume
% Structural Definitions
% Version 1.0 (19/09/2019)
%
% This template was created by:
% Vel (enquiries@latextypesetting.com)
% LaTeXTypesetting.com
%
%%%%%%%%%%%%%%%%%%%%%%%%%%%%%%%%%%%%%%%%%

%----------------------------------------------------------------------------------------
%	REQUIRED PACKAGES AND MISC CONFIGURATIONS
%----------------------------------------------------------------------------------------

\usepackage{graphicx} % Required for including images
\graphicspath{{Figures/}{./}} % Specifies where to look for included images (trailing slash required)

\setlength{\parindent}{0pt} % Suppress paragraph indentation

\usepackage{tabto} % Required for tabbing
\TabPositions{7cm, 10cm} % Specify fixed tab indents used by the \tab command

\usepackage{changepage} % Required for temporarily indenting text blocks

\raggedright % Left align all text (i.e. suppress justification)

\usepackage{lastpage} % Required to determine the total number of pages

\usepackage{moresize}

%----------------------------------------------------------------------------------------
%	MARGINS
%----------------------------------------------------------------------------------------

\usepackage[
	letterpaper, % Paper size
	top=2.5cm, % Top margin
	bottom=2.5cm, % Bottom margin
	inner=3cm, % Inner margin
	outer=3cm, % Outer margin
	footskip=1.4cm, % Space from the bottom margin to the baseline of the footer
	headsep=0.8cm, % Space from the top margin to the baseline of the header
	headheight=0.5cm, % Height of the header
	%showframe % Uncomment to show the frames around the margins for debugging purposes
]{geometry}

%----------------------------------------------------------------------------------------
%	FONTS
%----------------------------------------------------------------------------------------

%\usepackage[utf8]{inputenc} % Required for inputting international characters
%\usepackage[T1]{fontenc} % Output font encoding for international characters
%
%\usepackage{palatino} % Use the Palatino font as the body font
%\usepackage[extrabold]{raleway} % Use the extrabold weight of Raleway as the sans font

% To use the Copperplate Gothic Bold font, you will need to use XeLaTeX which will use it if it is installed in your operating system.
% Comment the lines above in this block and uncomment the lines below in order to switch to using XeLaTeX (read the comments carefully when uncommenting to know which of the 2 Copperplate lines to uncomment).
% You will need to compile the template with xelatex after doing this.

\usepackage{fontspec} % Required for specifying custom fonts

\setmainfont{TeX Gyre Pagella} % Sets Palatino as main font, must be installed on the system compiling the document

%\setsansfont{Copperplate} % Note: Copperplate Gothic Bold does not come with macOS so I can't test that this works, but Copperplate does come with macOS and works after uncommenting this line. DO NOT uncomment this line if using Copperplate Gothic Bold.

\setsansfont{CopprplGoth Hv BT} % Note: you may need to take out the \bfseries from the definition of \resumename as this font has bold as the regular variant of the font and doesn't have a bold variant. Do this only if you run into problems with using the font.

%----------------------------------------------------------------------------------------
%	HEADERS AND FOOTERS
%----------------------------------------------------------------------------------------

\usepackage{fancyhdr} % Required for customising headers and footers
\pagestyle{fancy} % Enable custom headers and footers

\renewcommand{\headrulewidth}{0pt} % Remove default top horizontal rule

\fancyhf{} % Clear default headers/footers

\fancyfoot[L]{\footnotesize Updated \updated} % Left side footer
\fancyfoot[R]{\small\thepage~of \pageref{LastPage}} % Right side footer

%----------------------------------------------------------------------------------------
%	SECTIONS
%----------------------------------------------------------------------------------------

\usepackage{titlesec} % Required for modifying sections

%------------------------------------------------

\titleformat
	{\section} % Section type being modified
	[block] % Shape type, can be: hang, block, display, runin, leftmargin, rightmargin, drop, wrap, frame
	{\LARGE\bfseries} % Format of the whole section
	{} % Format of the section label
	{0pt} % Space between the title and label
	{} % Code before the label
	[] % Code after the label

\titlespacing{\section}{0pt}{10pt}{0pt} % Spacing around section titles, the order is: left, before and after

%----------------------------------------------------------------------------------------
%	CUSTOM STYLING
%----------------------------------------------------------------------------------------

\newenvironment{indented}{ % Environment for indenting blocks of text on the left
	\begin{adjustwidth}{20pt}{0pt} % Specify left indent for indented sections
}{
	\end{adjustwidth}
}

\newcommand{\jobtitle}[1]{{\large #1}} % Style the job title

\newcommand{\resumename}[1]{{\centering\HUGE\sffamily{#1}\par}} % Style the resume name

%----------------------------------------------------------------------------------------
%	LISTS
%----------------------------------------------------------------------------------------

\usepackage{enumitem} % Required for list customisation

\setlist{itemsep=0pt, topsep=0pt, leftmargin=*} % Customise lists with no separating around them, inside them and no left margin

%----------------------------------------------------------------------------------------
%	LINKS
%----------------------------------------------------------------------------------------

\usepackage{hyperref} % Required for links

\hypersetup{
	colorlinks=false, % Suppress coloring of links
	hidelinks, % Hide the default boxes around links
	pdftitle=Kristoffer David Knigga,
	pdfauthor=Kristoffer Knigga,
	pdfstartview={XYZ 0 0 1.00},
}

%----------------------------------------------------------------------------------------
%	CUSTOM COMMANDS
%----------------------------------------------------------------------------------------

\newcommand{\updated}[1]{\renewcommand{\updated}{#1}}
